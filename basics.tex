\section{مبانی کامپیوتر}

\begin{itemize}
\item
تعریف IT , ICT
\item
تعریف کامپیوتر و معرفی کامپیوترهای شخصی PC
\item
تعریف سخت افزار و نرم افزار
\item
اجزای اصلی کامپیوترهای شخصی
\item
عملکرد کامپیوتر
\item
عوامل تاثیر گذار در کامپیوتر
\item
معرفی پردازنده
\item
معرفی انواع حافظه
\item
معرفی RAM,ROM
\item
معرفی وسایل ذخیره اطلاعات
\item
دستگاههای ورودی
\item
دستگاههای خروجی
\item
معرفی کلیدهای صفحه کلید
\item
معرفی کلیدهای ماوس
\item
دستگاههای ورودی/خروجی
\item
معرفی Portها
\item
معرفی سایر اجزای واحد سیستم
\item
تعریف نرم افزار
\item
معرفی سیستم عامل و نرم افزارهای کاربردی
\item
آشنایی با بعضی نرم افزارهای کاربردی
\item
قوانین مرتبط با فناوری اطلاعات
\item
قانون حق نشر
\item
نرم افزارهای دارای گواهینامه
\item
نرم افزارهای دارای گواهینامه
\item
نرم افزارهای اشتراکی/ رایگان/ منبع باز
\item
شبکه ها:
\item
معرفی شبکه های کامپیوتری
\item
مفاهیم شبکه
\item
معرفی Extranet , Intranet , Internet
\item
کاربردهای اینترنت
\item
تبادل داده ها
\item
Download , Upload
\item
روش های اتصال به اینترنت
\item
پیغام رسان فوری
\item
انتقال صوت از طریق اینترنت
\item
امنیت:
\item
محافظت از داده ها و دستگاه ها
\item
هویت/ تصدیق
\item
نام کاربری و رمز عبور
\item
روش های انتخاب کلمه عبور مناسب
\item
امنیت داده ها
\item
نسخه پشتیبان
\item
دیواره آتش
\item
برنامه های مخرب کامپیوتری
\item
معرفی ویروس و روش های ورود ویروس به کامپیوتر و محافظت در برابر ویروس ها
\item
موضوعات مرتبط با سلامت
\item
ارگونومیک، نور، وضعیت صحیح استقرار و روشهای استفاده صحیح از کامپیوتر
\item
محیط زیست
\item
بازیافت
\item
استفاده بهینه از انرژی

\end{itemize}